%% Sizes
\pgfmathsetmacro{\cardWidth}{6.3}
\pgfmathsetmacro{\cardHeight}{8.8}

\pgfmathsetmacro{\cardMargin}{0.25}
\pgfmathsetmacro{\valueCircleRadius}{0.08}
\pgfmathsetmacro{\costPrimaryBarLength}{0.6}
\pgfmathsetmacro{\costPrimaryBarHeight}{0.15}
\pgfmathsetmacro{\costSecondaryBarLength}{0.35}
\pgfmathsetmacro{\costSecondaryBarHeight}{0.06}
\pgfmathsetmacro{\costBarDifferenceDistance}{0.06}
\pgfmathsetmacro{\costBarDistances}{0.5}

\pgfmathsetmacro{\primaryAbilityBoxHeight}{1}

\pgfmathsetmacro{\opacity}{0.4}


% Shapes
\def\shapeCard{(0,0) rectangle (\cardWidth,\cardHeight)}
\def\shapeCardInternal{(\cardMargin,\cardMargin) rectangle (\cardWidth-\cardMargin,\cardHeight-\cardMargin)}

\tikzset{
	cardcorners/.style={
		rounded corners=0.2cm
	}
}

% Commands
\newcommand{\cardBorder}{
	\draw[lightgray,cardcorners] \shapeCard;
}

\newcommand{\InternalCardBorder}[1]{
	\draw[#1,cardcorners] \shapeCardInternal;
	% top
	\draw[white, fill=white] (0.5*\cardWidth-2*\valueCircleRadius-0.4*\cardMargin, \cardHeight-0.75*\cardMargin) rectangle (0.5*\cardWidth+2*\valueCircleRadius+0.4*\cardMargin, \cardHeight-1.25*\cardMargin);
	\draw[#1, thick, fill=#1, fill opacity=\opacity] (0.5*\cardWidth-0.5*\cardMargin, \cardHeight-\cardMargin) circle (\valueCircleRadius);
	\draw[#1, thick, fill=#1, fill opacity=\opacity] (0.5*\cardWidth+0.5*\cardMargin, \cardHeight-\cardMargin) circle (\valueCircleRadius);
	\draw[#1] (0.5*\cardWidth-2*\valueCircleRadius-0.4*\cardMargin, \cardHeight-0.75*\cardMargin) -- (0.5*\cardWidth-2*\valueCircleRadius-0.4*\cardMargin, \cardHeight-1.25*\cardMargin);
	\draw[#1] (0.5*\cardWidth+2*\valueCircleRadius+0.4*\cardMargin, \cardHeight-0.75*\cardMargin) -- (0.5*\cardWidth+2*\valueCircleRadius+0.4*\cardMargin, \cardHeight-1.25*\cardMargin);
	% right
	\draw[white, fill=white] (\cardWidth-0.8*\cardMargin, 0.5*\cardHeight-0.6*\cardMargin) rectangle (\cardWidth-1.2*\cardMargin, 0.5*\cardHeight+0.6*\cardMargin);
	\draw[#1, thick, fill=#1, fill opacity=\opacity] (\cardWidth-\cardMargin, 0.5*\cardHeight) circle (\valueCircleRadius);
	\draw[#1] (\cardWidth-0.8*\cardMargin, 0.5*\cardHeight-0.6*\cardMargin) -- (\cardWidth-1.2*\cardMargin, 0.5*\cardHeight-0.6*\cardMargin);
	\draw[#1] (\cardWidth-0.8*\cardMargin, 0.5*\cardHeight+0.6*\cardMargin) -- (\cardWidth-1.2*\cardMargin, 0.5*\cardHeight+0.6*\cardMargin);
	% bottom
	\draw[white, fill=white] (0.5*\cardWidth-0.3*\cardMargin, 0.75*\cardMargin) rectangle (0.5*\cardWidth+0.3*\cardMargin, 1.25*\cardMargin);
	\draw[#1] (0.5*\cardWidth-0.3*\cardMargin, 0.75*\cardMargin) -- (0.5*\cardWidth-0.3*\cardMargin, 1.25*\cardMargin);
	\draw[#1] (0.5*\cardWidth+0.3*\cardMargin, 0.75*\cardMargin) -- (0.5*\cardWidth+0.3*\cardMargin, 1.25*\cardMargin);
	% left
	\draw[white, fill=white] (0.8*\cardMargin, 0.5*\cardHeight-3*\valueCircleRadius-0.6*\cardMargin) rectangle (1.2*\cardMargin, 0.5*\cardHeight+3*\valueCircleRadius+0.6*\cardMargin);
	\draw[#1, thick, fill=#1, fill opacity=\opacity] (\cardMargin, 0.5*\cardHeight) circle (\valueCircleRadius);
	\draw[#1, thick, fill=#1, fill opacity=\opacity] (\cardMargin, 0.5*\cardHeight-\valueCircleRadius-0.6*\cardMargin) circle (\valueCircleRadius);
	\draw[#1, thick, fill=#1, fill opacity=\opacity] (\cardMargin, 0.5*\cardHeight+\valueCircleRadius+0.6*\cardMargin) circle (\valueCircleRadius);
	\draw[#1] (0.8*\cardMargin, 0.5*\cardHeight-3*\valueCircleRadius-0.6*\cardMargin) -- (1.2*\cardMargin, 0.5*\cardHeight-3*\valueCircleRadius-0.6*\cardMargin);
	\draw[#1] (0.8*\cardMargin, 0.5*\cardHeight+3*\valueCircleRadius+0.6*\cardMargin) -- (1.2*\cardMargin, 0.5*\cardHeight+3*\valueCircleRadius+0.6*\cardMargin);
}

\newcommand{\costBar}[2]{
	\begin{tikzpicture}
		\draw[#1, thick, fill=#1, fill opacity=\opacity]
			(-0.5*\costPrimaryBarLength, -0.5*\costPrimaryBarHeight) rectangle
			(0.5*\costPrimaryBarLength, 0.5*\costPrimaryBarHeight);
		\draw[#2, fill=#2, fill opacity=\opacity]
			(-0.5*\costSecondaryBarLength, -0.5*\costPrimaryBarHeight-\costBarDifferenceDistance) rectangle
			(0.5*\costSecondaryBarLength, -0.5*\costPrimaryBarHeight-\costBarDifferenceDistance-\costSecondaryBarHeight);
		\draw[#2, fill=#2, fill opacity=\opacity]
			(-0.5*\costSecondaryBarLength, +0.5*\costPrimaryBarHeight+\costBarDifferenceDistance) rectangle
			(0.5*\costSecondaryBarLength, +0.5*\costPrimaryBarHeight+\costBarDifferenceDistance+\costSecondaryBarHeight);
	\end{tikzpicture}
}

\newcommand{\costBars}[1]{
	\node [rotate=90] at (0.5*\cardWidth, 0.5*\cardHeight) {#1};
}

\newcommand{\leftPrimaryAbility}[2]{
	\draw[#1, thick, fill=#1, fill opacity=\opacity]
		(1.25*\cardMargin+0.5*\primaryAbilityBoxHeight, \cardMargin+0.5*\primaryAbilityBoxHeight) --
		(1.25*\cardMargin+1.5*\primaryAbilityBoxHeight, \cardMargin+0.5*\primaryAbilityBoxHeight) --
		(1.25*\cardMargin+1.5*\primaryAbilityBoxHeight, \cardHeight-\cardMargin-0.5*\primaryAbilityBoxHeight) --
		(1.25*\cardMargin+0.5*\primaryAbilityBoxHeight, \cardHeight-\cardMargin-0.5*\primaryAbilityBoxHeight) --
		(1.25*\cardMargin+0.5*\primaryAbilityBoxHeight, \cardMargin+0.5*\primaryAbilityBoxHeight);
		\node [rotate=-90] at (1.25*\cardMargin+\primaryAbilityBoxHeight, 0.5*\cardHeight) {#2};
}
\newcommand{\leftSecondaryAbility}[2]{
	\draw[#1]
		(1.5*\cardMargin+1.5*\primaryAbilityBoxHeight, 2*\cardMargin+\primaryAbilityBoxHeight) --
		(1.5*\cardMargin+2*\primaryAbilityBoxHeight, 4*\cardMargin+\primaryAbilityBoxHeight) --
		(1.5*\cardMargin+2*\primaryAbilityBoxHeight, \cardHeight-4*\cardMargin-\primaryAbilityBoxHeight) --
		(1.5*\cardMargin+1.5*\primaryAbilityBoxHeight, \cardHeight-2*\cardMargin-\primaryAbilityBoxHeight);
		\node [rotate=-90] at (1.5*\cardMargin+1.75*\primaryAbilityBoxHeight, 0.5*\cardHeight) {\scalebox{0.65}{#2}};
}

\newcommand{\rightPrimaryAbility}[2]{
	\draw[#1, thick, fill=#1, fill opacity=\opacity]
		(\cardWidth-1.25*\cardMargin-0.5*\primaryAbilityBoxHeight, \cardMargin+0.5*\primaryAbilityBoxHeight) --
		(\cardWidth-1.25*\cardMargin-1.5*\primaryAbilityBoxHeight, \cardMargin+0.5*\primaryAbilityBoxHeight) --
		(\cardWidth-1.25*\cardMargin-1.5*\primaryAbilityBoxHeight, \cardHeight-\cardMargin-0.5*\primaryAbilityBoxHeight) --
		(\cardWidth-1.25*\cardMargin-0.5*\primaryAbilityBoxHeight, \cardHeight-\cardMargin-0.5*\primaryAbilityBoxHeight) --
		(\cardWidth-1.25*\cardMargin-0.5*\primaryAbilityBoxHeight, \cardMargin+0.5*\primaryAbilityBoxHeight);
		\node [rotate=90] at (\cardWidth-1.25*\cardMargin-\primaryAbilityBoxHeight, 0.5*\cardHeight) {#2};
}
\newcommand{\rightSecondaryAbility}[2]{
	\draw[#1]
		(\cardWidth-1.5*\cardMargin-1.5*\primaryAbilityBoxHeight, 2*\cardMargin+\primaryAbilityBoxHeight) --
		(\cardWidth-1.5*\cardMargin-2*\primaryAbilityBoxHeight, 4*\cardMargin+\primaryAbilityBoxHeight) --
		(\cardWidth-1.5*\cardMargin-2*\primaryAbilityBoxHeight, \cardHeight-4*\cardMargin-\primaryAbilityBoxHeight) --
		(\cardWidth-1.5*\cardMargin-1.5*\primaryAbilityBoxHeight, \cardHeight-2*\cardMargin-\primaryAbilityBoxHeight);
		\node [rotate=90] at (\cardWidth-1.5*\cardMargin-1.75*\primaryAbilityBoxHeight, 0.5*\cardHeight) {\scalebox{0.65}{#2}};
}

\usetikzlibrary{positioning,calc,arrows.meta,shapes.geometric,fit}
% Sizes
\pgfmathsetmacro{\abilityIconHeight}{0.5}
\pgfmathsetmacro{\resourceCircleRadius}{0.35*\abilityIconHeight}


\newcommand{\emptyCard}{
	\begin{tikzpicture}
		\cardBorder
		\InternalCardBorder{black}
		\costBarFirst{black}{black}
		\costBarSecond{black}{black}
		\costBarThird{black}{black}
		\costBarFourth{black}{black}
		\leftPrimaryAbility{black}{}
		\leftSecondaryAbility{black}{}
		\rightPrimaryAbility{black}{}
		\rightSecondaryAbility{black}{}
	\end{tikzpicture}
}

\newcommand{\abilityTrade}[3]{
	\begin{tikzpicture}
		\draw[white, thick, fill=white] (-0.2-\resourceCircleRadius, 0) circle (\resourceCircleRadius);
		\draw[#1, thick, fill=#1, fill opacity=\opacity] (-0.2-\resourceCircleRadius, 0) circle (\resourceCircleRadius);
		\node at (0, 0) {\bf{:}};
		\foreach \n in {1, ..., #3}{
			\draw[white, thick, fill=white] (\n*0.2+\n*2*\resourceCircleRadius-\resourceCircleRadius, 0) circle (\resourceCircleRadius);
			\draw[#2, thick, fill=#2, fill opacity=\opacity] (\n*0.2+\n*2*\resourceCircleRadius-\resourceCircleRadius, 0) circle (\resourceCircleRadius);
		}
	\end{tikzpicture}
}

\newcommand{\abilityGainResoures}[2]{
	\begin{tikzpicture}
		\foreach \n in {1, ..., #2}{
			\draw[white, thick, fill=white] (\n*0.2+\n*2*\resourceCircleRadius-\resourceCircleRadius, 0) circle (\resourceCircleRadius);
			\draw[#1, thick, fill=#1, fill opacity=\opacity] (\n*0.2+\n*2*\resourceCircleRadius-\resourceCircleRadius, 0) circle (\resourceCircleRadius);
		}
	\end{tikzpicture}
}

\newcommand{\abilitySwap}{
	\begin{tikzpicture}
		\node[rotate=90] (leftCard) at (-0.65,0) {\scalebox{0.085}{\emptyCard}};
		\node[rotate=90] (rightCard) at (0.65,0) {\scalebox{0.085}{\emptyCard}};
		\draw [->] (leftCard) -- (rightCard);
		\draw [->] (rightCard) -- (leftCard);
	\end{tikzpicture}
}

\newcommand{\abilityActivate}[1]{
	\begin{tikzpicture}
		\node[rotate=90] at (0,0) {\scalebox{0.1}{\emptyCard}};
		\node[#1, thick, ellipse, draw, fit={(-0.065*\cardWidth,-0.02*\cardHeight)(0,-0.01*\cardHeight)(0,-0.03*\cardHeight)(0.065*\cardWidth,-0.02*\cardHeight)}, inner sep=0mm] {};
	\end{tikzpicture}
}

\newcommand{\abilityBuy}[1]{
	\begin{tikzpicture}
		\node[rotate=90] at (0,0) {\scalebox{0.1}{\emptyCard}};
		\node[#1, thick, ellipse, draw, fit={(-0.065*\cardWidth, 0)(0, 0.01*\cardHeight)(0, -0.01*\cardHeight)(0.065*\cardWidth, 0)}, inner sep=0mm] {};
	\end{tikzpicture}
}

\newcommand{\abilityDiscard}[2]{
	\begin{tikzpicture}
		\foreach \n in {1, ..., #1}{
			\node at (\n*0.125*\cardWidth-0.125*\cardWidth+0.03,0) {\scalebox{0.09}{\emptyCard}};
			\draw[white, line width=1mm] (\n*0.125*\cardWidth-0.125*\cardWidth+0.03*\n+0.05*\cardWidth, 0.05*\cardHeight) -- (\n*0.125*\cardWidth-0.125*\cardWidth+0.03*\n-0.05*\cardWidth, -0.05*\cardHeight);
			\draw[#2, line width=1mm, opacity=\opacity] (\n*0.125*\cardWidth-0.125*\cardWidth+0.03*\n+0.05*\cardWidth, 0.05*\cardHeight) -- (\n*0.125*\cardWidth-0.125*\cardWidth+0.03*\n-0.05*\cardWidth, -0.05*\cardHeight);
		}
	\end{tikzpicture}
}

